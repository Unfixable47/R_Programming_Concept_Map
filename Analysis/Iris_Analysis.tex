% Options for packages loaded elsewhere
\PassOptionsToPackage{unicode}{hyperref}
\PassOptionsToPackage{hyphens}{url}
%
\documentclass[
]{article}
\usepackage{amsmath,amssymb}
\usepackage{iftex}
\ifPDFTeX
  \usepackage[T1]{fontenc}
  \usepackage[utf8]{inputenc}
  \usepackage{textcomp} % provide euro and other symbols
\else % if luatex or xetex
  \usepackage{unicode-math} % this also loads fontspec
  \defaultfontfeatures{Scale=MatchLowercase}
  \defaultfontfeatures[\rmfamily]{Ligatures=TeX,Scale=1}
\fi
\usepackage{lmodern}
\ifPDFTeX\else
  % xetex/luatex font selection
\fi
% Use upquote if available, for straight quotes in verbatim environments
\IfFileExists{upquote.sty}{\usepackage{upquote}}{}
\IfFileExists{microtype.sty}{% use microtype if available
  \usepackage[]{microtype}
  \UseMicrotypeSet[protrusion]{basicmath} % disable protrusion for tt fonts
}{}
\makeatletter
\@ifundefined{KOMAClassName}{% if non-KOMA class
  \IfFileExists{parskip.sty}{%
    \usepackage{parskip}
  }{% else
    \setlength{\parindent}{0pt}
    \setlength{\parskip}{6pt plus 2pt minus 1pt}}
}{% if KOMA class
  \KOMAoptions{parskip=half}}
\makeatother
\usepackage{xcolor}
\usepackage[margin=1in]{geometry}
\usepackage{color}
\usepackage{fancyvrb}
\newcommand{\VerbBar}{|}
\newcommand{\VERB}{\Verb[commandchars=\\\{\}]}
\DefineVerbatimEnvironment{Highlighting}{Verbatim}{commandchars=\\\{\}}
% Add ',fontsize=\small' for more characters per line
\usepackage{framed}
\definecolor{shadecolor}{RGB}{248,248,248}
\newenvironment{Shaded}{\begin{snugshade}}{\end{snugshade}}
\newcommand{\AlertTok}[1]{\textcolor[rgb]{0.94,0.16,0.16}{#1}}
\newcommand{\AnnotationTok}[1]{\textcolor[rgb]{0.56,0.35,0.01}{\textbf{\textit{#1}}}}
\newcommand{\AttributeTok}[1]{\textcolor[rgb]{0.13,0.29,0.53}{#1}}
\newcommand{\BaseNTok}[1]{\textcolor[rgb]{0.00,0.00,0.81}{#1}}
\newcommand{\BuiltInTok}[1]{#1}
\newcommand{\CharTok}[1]{\textcolor[rgb]{0.31,0.60,0.02}{#1}}
\newcommand{\CommentTok}[1]{\textcolor[rgb]{0.56,0.35,0.01}{\textit{#1}}}
\newcommand{\CommentVarTok}[1]{\textcolor[rgb]{0.56,0.35,0.01}{\textbf{\textit{#1}}}}
\newcommand{\ConstantTok}[1]{\textcolor[rgb]{0.56,0.35,0.01}{#1}}
\newcommand{\ControlFlowTok}[1]{\textcolor[rgb]{0.13,0.29,0.53}{\textbf{#1}}}
\newcommand{\DataTypeTok}[1]{\textcolor[rgb]{0.13,0.29,0.53}{#1}}
\newcommand{\DecValTok}[1]{\textcolor[rgb]{0.00,0.00,0.81}{#1}}
\newcommand{\DocumentationTok}[1]{\textcolor[rgb]{0.56,0.35,0.01}{\textbf{\textit{#1}}}}
\newcommand{\ErrorTok}[1]{\textcolor[rgb]{0.64,0.00,0.00}{\textbf{#1}}}
\newcommand{\ExtensionTok}[1]{#1}
\newcommand{\FloatTok}[1]{\textcolor[rgb]{0.00,0.00,0.81}{#1}}
\newcommand{\FunctionTok}[1]{\textcolor[rgb]{0.13,0.29,0.53}{\textbf{#1}}}
\newcommand{\ImportTok}[1]{#1}
\newcommand{\InformationTok}[1]{\textcolor[rgb]{0.56,0.35,0.01}{\textbf{\textit{#1}}}}
\newcommand{\KeywordTok}[1]{\textcolor[rgb]{0.13,0.29,0.53}{\textbf{#1}}}
\newcommand{\NormalTok}[1]{#1}
\newcommand{\OperatorTok}[1]{\textcolor[rgb]{0.81,0.36,0.00}{\textbf{#1}}}
\newcommand{\OtherTok}[1]{\textcolor[rgb]{0.56,0.35,0.01}{#1}}
\newcommand{\PreprocessorTok}[1]{\textcolor[rgb]{0.56,0.35,0.01}{\textit{#1}}}
\newcommand{\RegionMarkerTok}[1]{#1}
\newcommand{\SpecialCharTok}[1]{\textcolor[rgb]{0.81,0.36,0.00}{\textbf{#1}}}
\newcommand{\SpecialStringTok}[1]{\textcolor[rgb]{0.31,0.60,0.02}{#1}}
\newcommand{\StringTok}[1]{\textcolor[rgb]{0.31,0.60,0.02}{#1}}
\newcommand{\VariableTok}[1]{\textcolor[rgb]{0.00,0.00,0.00}{#1}}
\newcommand{\VerbatimStringTok}[1]{\textcolor[rgb]{0.31,0.60,0.02}{#1}}
\newcommand{\WarningTok}[1]{\textcolor[rgb]{0.56,0.35,0.01}{\textbf{\textit{#1}}}}
\usepackage{graphicx}
\makeatletter
\def\maxwidth{\ifdim\Gin@nat@width>\linewidth\linewidth\else\Gin@nat@width\fi}
\def\maxheight{\ifdim\Gin@nat@height>\textheight\textheight\else\Gin@nat@height\fi}
\makeatother
% Scale images if necessary, so that they will not overflow the page
% margins by default, and it is still possible to overwrite the defaults
% using explicit options in \includegraphics[width, height, ...]{}
\setkeys{Gin}{width=\maxwidth,height=\maxheight,keepaspectratio}
% Set default figure placement to htbp
\makeatletter
\def\fps@figure{htbp}
\makeatother
\setlength{\emergencystretch}{3em} % prevent overfull lines
\providecommand{\tightlist}{%
  \setlength{\itemsep}{0pt}\setlength{\parskip}{0pt}}
\setcounter{secnumdepth}{-\maxdimen} % remove section numbering
\ifLuaTeX
  \usepackage{selnolig}  % disable illegal ligatures
\fi
\IfFileExists{bookmark.sty}{\usepackage{bookmark}}{\usepackage{hyperref}}
\IfFileExists{xurl.sty}{\usepackage{xurl}}{} % add URL line breaks if available
\urlstyle{same}
\hypersetup{
  pdftitle={Iris Analysis},
  pdfauthor={Vincent},
  hidelinks,
  pdfcreator={LaTeX via pandoc}}

\title{Iris Analysis}
\author{Vincent}
\date{2023-09-21}

\begin{document}
\maketitle

\hypertarget{question-1}{%
\subsection{Question 1}\label{question-1}}

\begin{enumerate}
\def\labelenumi{\arabic{enumi}.}
\tightlist
\item
  Provide a summary of the dataset, highlighting the mean, median, and
  range for each measurement (sepal length, sepal width, petal length,
  petal width) for each species.
\end{enumerate}

\begin{Shaded}
\begin{Highlighting}[]
\NormalTok{iris\_summary }\OtherTok{\textless{}{-}}\NormalTok{ iris }\SpecialCharTok{\%\textgreater{}\%}
  \FunctionTok{group\_by}\NormalTok{(Species) }\SpecialCharTok{\%\textgreater{}\%}
  \FunctionTok{summarize}\NormalTok{(}
    \AttributeTok{Mean\_Sepal.Length =} \FunctionTok{mean}\NormalTok{(Sepal.Length),}
    \AttributeTok{Median\_Sepal.Length =} \FunctionTok{median}\NormalTok{(Sepal.Length),}
    \AttributeTok{Min\_Sepal.Length =} \FunctionTok{min}\NormalTok{(Sepal.Length),}
    \AttributeTok{Max\_Sepal.Length =} \FunctionTok{max}\NormalTok{(Sepal.Length),}
    
    \AttributeTok{Mean\_Sepal.Width =} \FunctionTok{mean}\NormalTok{(Sepal.Width),}
    \AttributeTok{Median\_Sepal.Width =} \FunctionTok{median}\NormalTok{(Sepal.Width),}
    \AttributeTok{Min\_Sepal.Width =} \FunctionTok{min}\NormalTok{(Sepal.Width),}
    \AttributeTok{Max\_Sepal.Width =} \FunctionTok{max}\NormalTok{(Sepal.Width),}
    
    \AttributeTok{Mean\_Petal.Length =} \FunctionTok{mean}\NormalTok{(Petal.Length),}
    \AttributeTok{Median\_Petal.Length =} \FunctionTok{median}\NormalTok{(Petal.Length),}
    \AttributeTok{Min\_Petal.Length =} \FunctionTok{min}\NormalTok{(Petal.Length),}
    \AttributeTok{Max\_Petal.Length =} \FunctionTok{max}\NormalTok{(Petal.Length),}
    
    \AttributeTok{Mean\_Petal.Width =} \FunctionTok{mean}\NormalTok{(Petal.Width),}
    \AttributeTok{Median\_Petal.Width =} \FunctionTok{median}\NormalTok{(Petal.Width),}
    \AttributeTok{Min\_Petal.Width =} \FunctionTok{min}\NormalTok{(Petal.Width),}
    \AttributeTok{Max\_Petal.Width =} \FunctionTok{max}\NormalTok{(Petal.Width)}
\NormalTok{  )}
\end{Highlighting}
\end{Shaded}

\hypertarget{question-2-question-3}{%
\subsection{Question 2 \& Question 3}\label{question-2-question-3}}

\begin{enumerate}
\def\labelenumi{\arabic{enumi}.}
\setcounter{enumi}{1}
\item
  Plot the distribution of each measurement (sepal length, sepal width,
  petal length, petal width) using histograms. Color the histograms
  based on species.
\item
  Determine which species has the highest average petal length.
\end{enumerate}

\hypertarget{the-species-with-the-highest-average-petal-length-is-virginica}{%
\paragraph{The Species with the highest average Petal length is
Virginica}\label{the-species-with-the-highest-average-petal-length-is-virginica}}

\begin{verbatim}
## [[1]]
\end{verbatim}

\includegraphics{Iris_Analysis_files/figure-latex/data list-1.pdf}

\begin{verbatim}
## 
## [[2]]
\end{verbatim}

\includegraphics{Iris_Analysis_files/figure-latex/data list-2.pdf}

\begin{verbatim}
## 
## [[3]]
\end{verbatim}

\includegraphics{Iris_Analysis_files/figure-latex/data list-3.pdf}

\begin{verbatim}
## 
## [[4]]
\end{verbatim}

\includegraphics{Iris_Analysis_files/figure-latex/data list-4.pdf} \#\#
Question 4

\begin{enumerate}
\def\labelenumi{\arabic{enumi}.}
\setcounter{enumi}{3}
\tightlist
\item
  Identify any potential outliers in sepal width for each species using
  a boxplot.
\end{enumerate}

\hypertarget{within-the-iris-data-set-there-is-4-total-outliers-within-sepal-width.-two-outliers-within-the-setorsa-and-two-outliers-within-viriginica2}{%
\paragraph{Within the Iris data set there is 4 total outliers within
sepal width. Two outliers within the setorsa, and two outliers within
viriginica2}\label{within-the-iris-data-set-there-is-4-total-outliers-within-sepal-width.-two-outliers-within-the-setorsa-and-two-outliers-within-viriginica2}}

\includegraphics{Iris_Analysis_files/figure-latex/irisBoxplot-1.pdf}
\#\# Question 5

\begin{enumerate}
\def\labelenumi{\arabic{enumi}.}
\setcounter{enumi}{4}
\tightlist
\item
  Using a scatter plot, visualize the relationship between sepal length
  and petal length. Color the points based on species and fit a linear
  regression line to the plot.
\end{enumerate}

\includegraphics{Iris_Analysis_files/figure-latex/scatter plot-1.pdf}

\hypertarget{question-6}{%
\subsection{Question 6}\label{question-6}}

\begin{enumerate}
\def\labelenumi{\arabic{enumi}.}
\setcounter{enumi}{5}
\tightlist
\item
  Calculate the correlation coefficient between sepal length and petal
  length for each species.
\end{enumerate}

\begin{verbatim}
## # A tibble: 3 x 2
##   Species    correlation
##   <fct>            <dbl>
## 1 setosa           0.267
## 2 versicolor       0.754
## 3 virginica        0.864
\end{verbatim}

\end{document}
